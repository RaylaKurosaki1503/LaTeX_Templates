\chapter{Chapter 1}
\section{Section 1}
\begin{theorem}[Logic algebra]
    \label{th:logicalgebra}
    \index{logic algebra}
    \lipsum[1]
\end{theorem}

\begin{definition}[Rational Cauchy sequence]
    \label{th:rationalcauchysequence}
    \index{rational Cauchy sequence}
    \lipsum[1]
\end{definition}

\begin{note}
    \lipsum[1]
\end{note}

\begin{example}[Solving the equation $x^2 = 2$]
    \lipsum[1]
\end{example}

\begin{problem}
    \lipsum[1]\newline
    \begin{solution}
        \lipsum[1]
    \end{solution}
\end{problem}

\subsection{SubSection 1}
\lipsum[1]

\section{Section 2}
\begin{theorem}
    \lipsum[1]
\end{theorem}

\begin{definition}
    \lipsum[1]
\end{definition}

\begin{note}
    \lipsum[1]
\end{note}

\begin{example}
    \lipsum[1]
\end{example}

\begin{problem}
    \lipsum[1]\newline
    \begin{solution}
        \lipsum[1]
    \end{solution}
\end{problem}

\subsection{SubSection 2}
\lipsum[1]

\section{Section 3}
\begin{theorem}
    \lipsum[1]
\end{theorem}

\begin{definition}
    \lipsum[1]
\end{definition}

\begin{note}
    \lipsum[1]
\end{note}

\begin{example}
    \lipsum[1]
\end{example}

\begin{problem}
    \lipsum[1]\newline
    \begin{solution}
        \lipsum[1]
    \end{solution}
\end{problem}

\subsection{SubSection 3}
\lipsum[1]
