\chapter{Chapter 1}
\section{Section 1}
\begin{theorem}[Logic algebra]
    \label{th:logicalgebra}
    \index{logic algebra}
    Let $P$, $Q$ and $R$ be logical propositions (true or false). Then the following propositions are true:
    \small
    \begin{align*}
        P \land Q &\Leftrightarrow Q \land P & P \lor  Q &\Leftrightarrow Q \lor P  && \text{(commutative laws)}\\
        (P \land Q) \land R &\Leftrightarrow P \land (Q \land R) & (P \lor Q)  \lor  R &\Leftrightarrow P \lor  (Q \lor  R) && \text{(associative laws)}\\
        P \land (Q \lor  R) &\Leftrightarrow (P \land Q) \lor  (P \land R) & P \lor  (Q \land R) &\Leftrightarrow (P \lor  Q) \land (P \lor  R) && \text{(distributive laws)}\\
        \lnot (P \land Q) &\Leftrightarrow \lnot P \lor  \lnot Q & \lnot (P \lor  Q) &\Leftrightarrow \lnot P \land \lnot Q && \text{(De Morgan's laws)}
    \end{align*}
\end{theorem}

\begin{definition}[Rational Cauchy sequence]
    \label{th:rationalcauchysequence}
    \index{rational Cauchy sequence}
    A rational Cauchy sequence is a rational sequence $(x_n)_{n=0}^{\infty}$ such that
    \begin{equation}
        \forall \epsilon \in \mathbb{Q}_+ \; \exists N \in \mathbb{N} : m, n \geq N \Rightarrow |x_m - x_n| < \epsilon.
    \end{equation}
    In other words, for each (small) rational number $\epsilon > 0$ there is a (big) number $N$ such that the distance $|x_m - x_n|$ between $x_m$ and $x_n$ is less than $\epsilon$ if both $m$ and $n$ are larger than or equal to $N$.
\end{definition}

\begin{note}
\lipsum[1]
\end{note}

\begin{example}[Solving the equation $x^2 = 2$]
Consider the equation $x^2 = 2$. It is easy to prove that this equation does not have any rational solutions. However, consider the following iteration formula:
    \begin{equation}
        x_n = \frac{x_{n-1} + 2 / x_{n - 1}}{2},
    \end{equation}
    where $n = 1,2,3,\ldots$ and $x_0 = 1$. The resulting sequence of rational numbers quickly approaches a number in the vicinity of $x = 1.4142135623731$:
    \begin{displaymath}
        \begin{array}{rclcl}
            x_0 &=& 1 \\
            x_{1} &=& (x_{0} + 2 / x_{0}) / 2 &=& 1.5 \\
            x_{2} &=& (x_{1} + 2 / x_{1}) / 2 &\approx& 1.4166666666667 \\
            x_{3} &=& (x_{2} + 2 / x_{2}) / 2 &\approx& 1.4142156862745 \\
            x_{4} &=& (x_{3} + 2 / x_{3}) / 2 &\approx& 1.4142135623747 \\
            x_{5} &=& (x_{4} + 2 / x_{4}) / 2 &\approx& 1.4142135623731 \\
            x_{6} &=& (x_{5} + 2 / x_{5}) / 2 &\approx& 1.4142135623731 \\
            x_{7} &=& (x_{6} + 2 / x_{6}) / 2 &\approx& 1.4142135623731 \\
            x_{8} &=& (x_{7} + 2 / x_{7}) / 2 &\approx& 1.4142135623731 \\
            x_{9} &=& (x_{8} + 2 / x_{8}) / 2 &\approx& 1.4142135623731 \\
            x_{10} &=& (x_{9} + 2 / x_{9}) / 2 &\approx& 1.4142135623731
        \end{array}
    \end{displaymath}
    We will later see that this iteration, or any other equivalent iteration, defines the real number $\sqrt{2}$.
\end{example}

\begin{problem}
Interpret the following set definition (Russell's paradox) and discuss whether $X \in X$ or $X \notin X$:
    \begin{equation}
        X = \{x \mid x \notin x\}.
    \end{equation}

\begin{solution}
\lipsum[1]
\end{solution}
\end{problem}

\subsection{SubSection 1}
\Blindtext

\section{Section 2}
\Blindtext

\subsection{SubSection 2}
\Blindtext

\section{Section 3}
\Blindtext

\subsection{SubSection 3}
\Blindtext