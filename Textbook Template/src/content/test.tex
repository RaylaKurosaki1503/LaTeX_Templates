\chapter{First chapter}
%
%\begin{summary}
%  This first chapter illustrates how to use various elements of this
%  text book template, such as definitions, theorems and exercises. You
%  may want to start each chapter with a meta summary like this one, to
%  explain to the reader what the chapter is all about, why it is
%  important and how it fits into the bigger picture of the
%  book. Another useful tip is to put the contents of each chapter into
%  a separate \LaTeX{} file and then use the command
%  \texttt{\textbackslash{}input\{\}} to include the chapter in the
%  main document.
%\end{summary}
%
\section{First section}
%
Let's start out with the following theorem.
%
\begin{theorem}[Logic algebra]
    \label{th:logicalgebra}
    \index{logic algebra}
    \lipsum[1]
\end{theorem}
\begin{proof}
    \lipsum[1]
\end{proof}

\section{Second section}

We begin our next section with the following central definition.

\begin{definition}[Rational Cauchy sequence]
    \label{th:rationalcauchysequence}
    \index{rational Cauchy sequence}
    \lipsum[1]
\end{definition}

\begin{note}
    \lipsum[1]
\end{note}

\begin{example}[Solving the equation $x^2 = 2$]
    \lipsum[1]
\end{example}

\section{Third section}

Now let's move on to the definition of the real number system. This
may be defined in a multitude of ways, one of which is to think about
a real number as a rational Cauchy sequence, or rather the equivalence
class of Cauchy sequences ``converging to'' that number.

\begin{definition}[The real numbers]
    \index{real numbers}
    \lipsum[1]
\end{definition}

Now that this is settled, lets prove the completeness of the real
number system.

\begin{theorem}[The completeness of the real numbers]
    \label{th:realnumberscomplete}
    \index{completeness of the real numbers}
    \lipsum[1]
\end{theorem}
\begin{proof}
    \lipsum[1]
\end{proof}

For further reading, there are several excellent works that one could
cite, such as~\cite{book:2946356}.

\begin{example}
    \lipsum[1]
\end{example}
\begin{solution}
    \lipsum[1]
\end{solution}

\begin{example}
    \lipsum[1]
\end{example}
\begin{solution}
    \lipsum[1]
\end{solution}

\begin{example}
    \lipsum[1]
\end{example}
\begin{solution}
    \lipsum[1]
\end{solution}

\begin{problem}
    \lipsum[1]
\end{problem}
\begin{solution}
    \lipsum[1]
\end{solution}